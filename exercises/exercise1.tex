% \PassOptionsToPackage{draft}{graphicx}
\documentclass[./../main.tex]{subfiles}

\graphicspath{{img/}}

\begin{document}
    \begin{exercise}
        Para la reacción de \ch{^{48} Ca} a \qty{215}{\MeV} (energía cinética en el sistema de laboratorio) con \ch{^{208} Pb} ángulo de \qty{20}{\degree}.

        \begin{enumerate}[label = \alph*)]
            \item Calcular la altura de la barrera de Coulomb. Expresar el resultado en \si{\MeV}.
            
            \begin{solution}
                Sabemos que la barrera de Coulomb está dada por

                \begin{equation}
                    V_{C} = \dfrac{Z_{p}Z_{t}\e^{2}}{R_{t} + r_{p}},
                    \label{eq:CoulombBarrier}
                \end{equation}

                donde \(Z_{p}\) pertenece al blanco, \(Z_{2}\) al proyectil, \(\e\) es la carga del electrón, \(R_{t}\) es el radio del blanco y \(r_{p}\) es el radio del proyectil.

                Para este caso tenemos que \(Z_{p} = 20\) para el \ch{^{48} Ca} y \(Z_{t} = 82\) para el \ch{^{208} Pb}, cuyos respectivos radios se obtienen a partir de la expresión \(R = \qty{1.2}{\fm} \cdot A^{1/3}\). Por lo cual,

                \begin{empheq}[box = \fbox]{align*}
                    R_{t} &= \qty{7.11}{\fm},\\
                    r_{p} &= \qty{4.36}{\fm}.
                \end{empheq}

                Y, además, que \(\e^{2} = \qty{1.44}{\MeV \fm}\).

                Sustituyendo estos valores en \cref{eq:CoulombBarrier} obtenemos que la altura de la barrera de Coulomb es de

                \begin{align*}
                    V_{C} &= \dfrac{20 \cdot 82 \cdot \qty{1.44}{\MeV \fm}}{\qty{7.11}{\fm} + \qty{4.36}{\fm}}\\
                    \Acolorboxed{V_{C} &= \qty{205.89}{\MeV}.}
                \end{align*}
            \end{solution}
            
            \item Calcular el parámetro de Sommerfield (\(\eta\)) y diga el tipo de dispersión elástica que ocurre.
            
            \item Calcular la sección eficaz diferencial de Rutherford. Exprese su resultado en milibarn (\si{\mb}).
        \end{enumerate}
    \end{exercise}
\end{document}