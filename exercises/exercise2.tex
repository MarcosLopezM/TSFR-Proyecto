\documentclass[../main]{subfiles}

\begin{document}
    \begin{exercise}
        Por otro lado, se puede ilustrar que la materia a nivel microscópico es hueva. Hacer el siguiente cálculo:

        Suponer que se tiene un balín esférico de radio \(r = \SI{1}{\cm}\) compuesto de \ch{^{nat} Fe} (con \(A = 54\) (\SI{6}{\percent}), \(A = 56\) (\SI{92}{\percent}) y \(A = 57\) (\SI{2}{\percent}), isótopos más abundantes del \ch{Fe}). Calcular el volumen de un núcleo de \ch{Fe} cuyo radio es \(r = r_{0}A^{\sfrac{1}{3}}\). Suponiendo que no hay repulsión coulombiana, ¿cuántos átomos de \ch{Fe} cabrían en el balín de \SI{1}{\cm} de radio? Calcular en [\unit{\kg}] lo que pesaría el balín con esa cantidad de átomos.

        \textbf{NOTA}: Tomen el valor de \(r_{0}\) con las unidades convenientes.

        \begin{solution}
            
        \end{solution}
    \end{exercise}
\end{document}