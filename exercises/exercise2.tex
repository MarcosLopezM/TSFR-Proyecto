\documentclass[./../main.tex]{subfiles}
\graphicspath{{img/}}

\begin{document}
    \begin{exercise}[Perturbación en un sistema de tres estados (valor total: 5 pt)]
        Considera un sistema que solo tiene tres estados linealmente independientes. Considera ahora que el Hamiltoniano del sistema está dado por la siguiente matriz:

        \begin{equation*}
            \observable{H} = 
                V_{0}
                \matrice{1 - \epsilon, 0, 0, 0, 1, \epsilon, 0, \epsilon, 2},
        \end{equation*}

        en donde \(V_{0}\) es una constante y \(\epsilon\) es un número pequeño, es decir, \(\epsilon \ll 1\).

        \begin{enumerate}[label=(\alph*)]
            \item Valor: 0.5 pt - Si consideramos que \(\epsilon\) es ele resultado de una perturbación, escribe este Hamiltoniano como la suma de un Hamiltoniano imperturbado \(\observable{H}[0]\) y una perturbación \(\observable{W}\), es decir, \(\observable{H} = \observable{H}[0] + \observable{W}\), de tal forma que \(\observable{H} = \observable{H}[0]\) cuando \(\epsilon = 0\).
            
            \begin{solution}
                Para lograr escribir \(\observable{H}\) como \(\observable{H}[0] + \observable{W}\), recordamos que \(\observable{H}[0]\) se obtiene cuando \(\epsilon = 0\), por lo que \(\observable{W}\) debe estar únicamente en términos de \(\epsilon\). Así,

                \begin{equation}
                    \observable{H}[0] = V_{0}
                    \matrice{1, 0, 0, 0, 1, 0, 0, 0, 2}.
                    \label{eq:unperturbated-hamiltonian}
                \end{equation}

                Y, \(\observable{W}\),

                \begin{equation}
                    \observable{W} = V_{0} \epsilon\matrice{-1, 0, 0, 0, 0, 1, 0, 1, 0}.
                    \label{eq:perturbation-operator}
                \end{equation}

                El Hamiltoniano \(\observable{H}\) queda entonces como

                \begin{empheq}[box = \color{pinkwave}\fbox]{equation}
                    \observable{H} = V_{0}\matrice{1, 0, 0, 0, 1, 0, 0, 0, 2} + V_{0} \epsilon\matrice{-1, 0, 0, 0, 0, 1, 0, 1, 0}.
                    \label{eq:total-hamiltonian}
                \end{empheq}
            \end{solution}
            
            \pagebreak
            \item Valor: 0.5 pt - ¿Quiénes son los eigenvalores y los eigenvectores del Hamiltoniano imperturbado \(\observable{H}[0]\)? Nota que este Hamiltoniano tiene un eigenvalor no degenerado y dos eigenvalores degenerados.
            
            \begin{solution}
                Inmediatamente de \cref{eq:unperturbated-hamiltonian} notamos que los eigenvalores de \(\observable{H}[0]\) son
                
                \begin{empheq}[box = \color{pinkwave}\fbox]{align*}
                    \overline{\lambda}_{1} &= V_{0},\\
                    \lambda_{2} &= 2V_{0},
                \end{empheq}
                
                con \(\overline{\lambda}_{1}\) doblemente degenerado. Y que sus eigenvectores son

                \begin{empheq}[box = \color{pinkwave}\fbox]{equation*}
                    \ket{\psi_{1}} = \matrice[1]{1, 0, 0};\quad \ket{\psi_{2}} = \matrice[1]{0, 1, 0};\quad \ket{\psi_{3}} = \matrice[1]{0, 0, 1}. 
                \end{empheq}
            \end{solution}
            
            \item Valor: 1.0pt - El Hamiltoniano \(\observable{H}\) puede resolverse exactamente Encuentra los eigenvalores exactos de \(\observable{H}\). Una vez que los hayas encontrado, exprésalos como una serie de potencias de Taylor en \(\epsilon\) hasta segundo orden, es decir, conserva todos los términos con orden igual o menor a \(\epsilon^{2}\).
            
            \begin{solution}
                Reescribimos \(\observable{H}\) para no estar cargando con coeficientes, tal que,

                \begin{align*}
                    \observable{H}[][\prime] &= \frac{1}{V_{0}}\observable{H},\\
                    \observable{H}[][\prime] &= \matrice{1 - \epsilon, 0, 0, 0, 1, \epsilon, 0, \epsilon, 2}.
                \end{align*}

                Resolvemos el problema de eigenvalores, tal que el polinomio característico de \(\observable{H}[][\prime]\) es

                \begin{equation*}
                    [(1 - \epsilon) - \lambda][(1 - \lambda)(2 - \lambda) - \epsilon^{2}] = 0.
                \end{equation*}

                Los eigenvalores son

                \begin{empheq}[box = \color{customBlue}\fbox]{align*}
                    \lambda_{1} &= 1 - \epsilon,\\
                    \lambda_{2} &= \frac{1}{2}\left(3 + \sqrt{1 + 4\epsilon^{2}}\right),\\
                    \lambda_{2} &= \frac{1}{2}\left(3 - \sqrt{1 + 4\epsilon^{2}}\right).
                \end{empheq}

                Haciendo la expansión de Taylor correspondiente para cada uno de los eigenvalores tenemos que

                \begin{empheq}[box = \color{customBlue}\fbox]{align*}
                    \lambda_{1} &= 1 - \epsilon,\\
                    \lambda_{2} &\simeq 2 + \epsilon^{2},\\
                    \lambda_{3} &\simeq 1 - \epsilon^{2}.
                \end{empheq}

                Multiplicando por \(V_{0}\) tenemos que los eigenvalores de \(\observable{H}\) son

                \begin{empheq}[box = \color{pinkwave}\fbox]{equation}
                    \begin{alignedat}{1}
                        \omega_{1} &= V_{0}(1 - \epsilon),\\
                        \omega_{2} &\simeq V_{0}(2 - \epsilon^{2}),\\
                        \omega_{3} &\simeq V_{0}(1 - \epsilon^{2}).
                    \end{alignedat}
                    \label{eq:hamiltonian-eigenvalues}
                \end{empheq}
            \end{solution}
            
            \item Valor: 1.0pt - Utiliza el caso \textbf{no} degenerado de la teoría de perturbaciones independientes del tiempo para calcular las correcciones a primer orden y segundo orden del eigenvalor \textbf{no} degenerado del Hamiltoniano imperturbado \(\observable{H}[0]\). Compara este resultado con el que encontraste en (c).
            
            \begin{solution}
                Sabemos que el eigenvalor no degenerado es \(\lambda_{3} = 2V_{0}\). Recordamos entonces que la corrección a primer orden viene dada por

                \begin{equation*}
                    E_{3}^{(1)} = \matrixel{\psi_{3}^{(0)}}{\observable{W}}{\psi_{3}^{(0)}}.
                \end{equation*}

                Así,

                \begin{align*}
                    E_{3}^{(1)} &= V_{0} \epsilon \begin{pmatrix}
                        0 & 0 & 1
                    \end{pmatrix}\matrice{-1, 0, 0, 0, 0, 1, 0, 1, 0}\matrice[1]{0, 0, 1},\\
                    &= V_{0} \epsilon \begin{pmatrix}
                        0 & 0 & 1
                    \end{pmatrix}\matrice[1]{0, 1, 0},\\
                    \Acolorboxed[customBlue]{E_{3}^{(1)} &= 0.}
                \end{align*}

                Es decir, no hay corrección de primer orden.

                \pagebreak
                La corrección de segundo orden para el caso no degenerado se obtiene a partir de

                \begin{equation*}
                    E_{n}^{(2)} = \sum_{k \neq n} \dfrac{\abs{\matrixel{\psi_{n}^{(0)}}{\observable{W}}{\psi_{n}^{(0)}}}^{2}}{E_{n}^{(0)} - E_{k}^{(0)}}.
                \end{equation*}

                Así,
                
                \begin{equation}
                    E_{3}^{2} = \dfrac{\abs{\matrixel{\psi_{1}^{(0)}}{\observable{W}}{\psi_{3}^{(0)}}}^{2}}{E_{3}^{(0)} - E_{1}^{(0)}} + \dfrac{\abs{\matrixel{\psi_{2}^{(0)}}{\observable{W}}{\psi_{3}^{(0)}}}^{2}}{E_{3}^{(0)} - E_{2}^{(0)}}.
                    \label{eq:non-degeneracy-second-order-correction}
                \end{equation}

                Por un lado tenemos que

                \begin{align*}
                    \matrixel{\psi_{1}^{(0)}}{\observable{W}}{\psi_{3}^{(0)}} &= V_{0} \epsilon \begin{pmatrix}
                        1 & 0 & 0
                    \end{pmatrix}\matrice{-1, 0, 0, 0, 0, 1, 0, 1, 0}\matrice[1]{0, 0, 1},\\
                    \Aboxed{\abs{\matrixel{\psi_{1}^{(0)}}{\observable{W}}{\psi_{3}^{(0)}}}^{2} &= 0.}
                \end{align*}

                Por el otro,

                \begin{align*}
                    \matrixel{\psi_{2}^{(0)}}{\observable{W}}{\psi_{3}^{(0)}} &= V_{0} \epsilon \begin{pmatrix}
                        0 & 1 & 0
                    \end{pmatrix}\matrice{-1, 0, 0, 0, 0, 1, 0, 1, 0}\matrice[1]{0, 0, 1},\\
                    &= V_{0} \epsilon,\\
                    \Aboxed{\abs{\matrixel{\psi_{2}^{(0)}}{\observable{W}}{\psi_{3}^{(0)}}}^{2} &= V_{0}^{2} \epsilon^{2}.}
                \end{align*}

                Y además sabemos que \(E_{3}^{(0)} = 2V_{0}\) y \(E_{2}^{(0)} = V_{0}\), entonces

                \begin{align*}
                    E_{3}^{(0)} - E_{2}^{(0)} &= 2V_{0} - V_{0},\\
                    E_{3}^{(0)} - E_{2}^{(0)} &= V_{0}.
                \end{align*}

                Sustituyendo lo anterior en \cref{eq:non-degeneracy-second-order-correction},

                \begin{align*}
                    E_{3}^{(2)} &= \dfrac{V_{0}^{2} \epsilon^{2}}{V_{0}},\\
                    E_{3}^{(2)} &= V_{0} \epsilon^{2}.
                \end{align*}

                Por lo que la corrección de la energía a segundo orden es:
                
                \begin{align*}
                    E_{3} &\simeq E_{3}^{(0)} + E_{3}^{(1)} + E_{3}^{(2)},\\
                    &\simeq 2V_{0} + 0 + V_{0} \epsilon^{2},\\
                    E_{3} &\simeq V_{0}(2 + \epsilon^{2}).
                \end{align*}

                Comparando el valor de \(E_{3}\) con el resultado obtenido en \cref{eq:hamiltonian-eigenvalues} observamos que es lo mismo, \idest

                \begin{empheq}[box = \color{pinkwave}\fbox]{equation*}
                    E_{3} \simeq V_{0}(2 + \epsilon^{2}) = V_{0}(2 + \epsilon^{2}) \simeq \omega_{3}.
                \end{empheq}
            \end{solution}
            
            \item Valor: 2.0pt - Utiliza ahora el caso degenerado de la teoría de perturbaciones independientes del tiempo para calcular las correcciones a primer orden de los dos eigenvalores degenerados de \(\observable{H}[0]\). Compara con los resultados del inciso (c).
            
            \begin{solution}
                Recordamos que la corrección primer orden de la energía para el caso degenerado viene dada por

                \begin{equation}
                    E_{\pm}^{(1)} = \frac{1}{2}\left[W_{aa} + W_{bb} \pm \sqrt{(W_{aa} - W_{bb})^{2} - 4\abs{W_{ab}}^{2}}\right],
                    \label{eq:degeneracy-energy-correction-first-order}
                \end{equation}

                donde

                \begin{equation*}
                    W_{ij} = \matrixel{\psi_{i}^{(0)}}{\observable{W}}{\psi_{j}^{(a)}}.
                \end{equation*}

                Debemos obtener \(W_{aa},\ W_{bb}\) y \(W_{ab}\),

                \begin{align*}
                    W_{aa} &= \matrixel{\psi_{a}^{(0)}}{\observable{W}}{\psi_{a}^{(0)}},\\
                    W_{bb} &= \matrixel{\psi_{b}^{(0)}}{\observable{W}}{\psi_{b}^{(0)}},\\
                    W_{ab} &= \matrixel{\psi_{a}^{(0)}}{\observable{W}}{\psi_{b}^{(0)}},
                \end{align*}

            donde \(\ket{\psi_{a}^{(0)}} = \ket{\psi_{1}}\) y \(\ket{\psi_{b}^{(0)}} = \ket{\psi_{2}}\).

            \pagebreak
            Así,

            \begin{align}
                W_{aa} &= V_{0} \epsilon \begin{pmatrix}
                    1 & 0 & 0
                \end{pmatrix} \matrice{-1, 0, 0, 0, 0, 1, 0, 1, 0}\matrice[1]{1, 0, 0},\nonumber\\
                &= V_{0} \epsilon \begin{pmatrix}
                    1 & 0 & 0
                \end{pmatrix}\matrice[1]{-1, 0, 0},\nonumber\\
                \Acolorboxed{W_{aa} &= -V_{0} \epsilon.}\label{eq:element-Waa}
            \end{align}

            \begin{align}
                W_{bb} &= V_{0} \epsilon \begin{pmatrix}
                    0 & 1 & 0
                \end{pmatrix} \matrice{-1, 0, 0, 0, 0, 1, 0, 1, 0}\matrice[1]{0, 1, 0},\nonumber\\
                &= V_{0} \epsilon \begin{pmatrix}
                    0 & 1 & 0
                \end{pmatrix}\matrice[1]{0, 0, 0},\nonumber\\
                \Acolorboxed{W_{aa} &= 0.}\label{eq:element-Wbb}
            \end{align}

            \begin{align}
                W_{ab} &= V_{0} \epsilon \begin{pmatrix}
                    1 & 0 & 0
                \end{pmatrix} \matrice{-1, 0, 0, 0, 0, 1, 0, 1, 0}\matrice[1]{0, 1, 0},\nonumber\\
                &= V_{0} \epsilon \begin{pmatrix}
                    1 & 0 & 0
                \end{pmatrix}\matrice[1]{0, 0, 1},\nonumber\\
                \Acolorboxed{W_{aa} &= 0.}\label{eq:element-Wab}
            \end{align}

            Sustituyendo \crefrange{eq:element-Waa}{eq:element-Wab} en \cref{eq:degeneracy-energy-correction-first-order},

            \begin{align*}
                E_{\pm}^{(1)} &= \frac{1}{2}\left[-V_{0} \epsilon \pm V_{0} \epsilon\right],\\
                \Acolorboxed[customBlue]{E_{\pm}^{(1)} &= \set{0, -V_{0}\epsilon}.}
            \end{align*}

            \pagebreak
            Por lo que la corrección de la energía \(E^{(1)}\) es:

            \begin{align*}
                E_{1} &\simeq E_{0}^{(1)} + E_{1}^{(1)},\\
                &\simeq V_{0} + (0 - V_{0}\epsilon),\\
                \Acolorboxed{E_{1} &\simeq V_{0}(1 - \epsilon).}
            \end{align*}

            Comparando con \cref{eq:hamiltonian-eigenvalues}, verificamos que el resultado es igual al obtenido en el inciso (c).
            \end{solution}
        \end{enumerate}
    \end{exercise}
\end{document}