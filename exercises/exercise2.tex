\documentclass[./../main.tex]{subfiles}
\graphicspath{{img/}}

\begin{document}
    \begin{exercise}
        Calcular en la tercera columna la sección de dispersión de Rutherford. Se anexa la ecuación para el cálculo en el sistema de laboratorio de energías de protones: \qtylist[list-units=single]{800;1000;1500}{\keV}. Graficar la sección experimental en unidades de \unit{\mb} y en la misma gráfica, la sección de RBS de la 3\textsuperscript{a} columna, consideren las unidades de energía que van a usar para realizar los cálculos, porque de otra forma, tendrán que usar dos gráficas o dos ejes Y.

        \begin{equation*}
            \sigma_{R}(E, \theta)\left[\unit{\mb\per\steradian}\right] = (\num{5.18e6}) \left[\dfrac{Z_{1}Z_{2}}{E\left[\unit{\keV}\right]}\right]^{2} \dfrac{\left\lbrace\sqrt{M_{2}^{2} - M_{1}^{2}(\sin\theta)^{2}} + M_{2}\cos\theta\right\rbrace^{2}}{M_{2}(\sin\theta)^{4}\left(\sqrt{M_{2}^{2} - M_{1}^{2}(\sin\theta)^{2}}\right)}.
        \end{equation*}

        La sección de Rutherford se puede calcular debido a que tanto el proyectil como el núcleo blanco se consideran como si fueran esferas con carga positiva y la fuerza es conocida: coulombiana.

        Clásicamente se podría esperar que la sección experimental sería dada por la ecuación de Rutherford, porque los protones no tienen la energía cinética requerida para lograr acercarse al núcleo y por lo tanto no actuarían las fuerzas nucleares.

        La comparación en las gráficas de las secciones muestran que la sección de dispersión experimental es mayor que la sección de Rutherford. Este por un efecto cuántico, denominado efecto túnel.

        \begin{figure}[htb]
            \centering
            \includegraphics[scale=0.8]{example-image-a}
        \end{figure}
    \end{exercise}
\end{document}